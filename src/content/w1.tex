\section{W1: Projects}

\textbf{Understand key elements of a project and why organisations use them.}

\textbf{Project}: temporary endeavour to create a unique product, service or outcome. It has some OBJECTIVE, introduce CHANGE to organisation, TEMPORARY (has a start and end date), UNIQUE (never been done before), CROSS-FUNCTIONAL (involves people from different departments), deals with the UNKNOWN, requires RESOURCES like people, time, money, equipment.

\textbf{Software project}: temporary endeavour as collaborative effort to plan, develop, test and deploy software that meets specific requirements within a defined timeframe and budget. It is GOAL ORIENTED (aim to create a new or CHANGE to existing), TIME CONSTRAINED, COLLABORATIVE, UNKNOWN, UNIQUE, RESOURCES.

Organisation use projects to (1) provide strategic alignment of key activities and visibility at the appropriate levels, (2) allows organisations to deliver change in a structured and formal manner outside of BAU (Business As Usual), (3) effective and efficient management of organisations limited resources (time, money, people, equipment), (4) establish ownership and accountability, (5) provide clarity, buy-in and agreement across what will be done, when, who, why and the outcomes.

\textbf{Understand the foundational components of project management.}

\textbf{Project management}: is the planning, delegating, monitoring and controlling of all aspects of a project, and motivating those involved to achieve the project objectives within the expected targets for time, costs, quality, scope, benefits and risks. Values lie in (1) organising and structuring scarce resources, (2) managing risk, (3) identifying and clearing issues, (4) managing and implementing change, (5) retaining and reusing knowledge, (6) organisational wide learning from past success and failures.

\textbf{Understand key skills and responsibilities/activities of a project manager.}
Project manager plans, organises, leads and controls. Effective communicators. Comfortable with change and complexity in changing environments. Action oriented.

\textbf{Understand the Project Initialization process, Business Case structure and why organisations use them.}

Business case contains executive summary, reasons for why it is required, business options, expected benefits, expected dis-benefits, timescale, costs, investment appraisal, major risks.

\textbf{Explore various investment techniques and financial models.}

ROI = (total discounted benefits - total discounted costs) / total discounted costs. Higher ROI is better.

NPV (net present value) = sum of all discounted cash flows - initial investment. Higher NPV is better.

Payback period = time taken to recover initial investment. Lower payback period is better.

ROM (rough order of magnitude) = initial estimate of costs. Used to determine if project is worth pursuing.

\textbf{Understand what a Project Charter is and how it is used.}

Project charter is a document that formally authorises a project. It gives the project manager the authority to apply organisational resources to project activities. It is used to ensure that all stakeholders understand the project, its objectives, scope, constraints, assumptions, risks, roles and responsibilities.

% \textbf{Authentication} is the process of showing an identity is real and genuine.\\
% \textbf{Three methods of authentication:} something you \textbf{know} (password), something you \textbf{have} (WebAuthn passkeys, public and private keys), something you \textbf{are} (biometrics like fingerprint).\\
% \includegraphics[width=\linewidth]{figs/password-entropy.png}
